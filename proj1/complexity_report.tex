\documentclass{article}
\usepackage[utf8]{inputenc}
\usepackage[left=3cm, right=3cm]{geometry}

\title{Análise da complexidade da solução para o problema das pessoas fundamentais}
\author{Patrick Fernandes\\81191 \and João Abrantes\\81845}
\date{Grupo 70}

\begin{document}
\maketitle
\section{Introdução}
O problema proposto consistia em, dado um certo número de pessoas presentes numa certa rede social e as ligações (bidirecionais) entre elas, que pessoas eram fundamentais para a difusão total de uma mensagem, ou seja, em que pessoas a informação tinha obrigatoriamente de passar para poder chegar a todas as pessoas (mencionadas) da rede social.\\
Pela natureza do problema, é possível concluir que a rede social pode ser descrita por um grafo não dirigido, em que as pessoas são representadas por vértices e as ligações entre pessoas são as arestas do grafo. Então, dada esta descrição do contexto, o problema pode ser reduzido a um em termos do grafo: que vértices são articulações do grafo, ou por outras palavras, que vértices, quando retirados, separariam o grafo original em dois grafos distintos.
\section{A solução}
\end{document}
