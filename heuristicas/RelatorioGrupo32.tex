\documentclass{article}
\usepackage[utf8]{inputenc}

\begin{document}
\section*{Análise Heurística do Modelo do Grupo XX}
Aqui se apresenta uma análise ao protótipo de papel, identificando erros com base nas 10 heuristicas de Nielson e propondo possiveis soluções.\\
\\
\textbf{Problema 1:} Não é percetivel se é percetivel quias os items do menu principal\\
\textbf{Heurística:} H2-1: Tornar o sistema visivel\\
\textbf{Descrição:} Quando à chegada do menu principal, aparece apenas uma opção de escolha, tendo que arrastar o dedo para a esquerda ou direita para identificar outras opções.\\
\textbf{Correção:} Criar um menu principal com as operações principais que o utilizador pode efectuar, como por exemplo, um menu em grelha.\\
\textbf{Severidade:} 3\\
\\
\textbf{Problema 2:} Imposibilidade de cancelar opções demoradas \\
\textbf{Heurística:} H2-3: Utilizador controla e exerce livre arbitrio\\
\textbf{Descrição:} Não é possivel visualizar o tempo que demora ao pedido para ser servido, não havendo, consequentemente, uma opção que permita cancelar o pedido caso o cliente deseje.\\
\textbf{Correção:} Criar uma opção que permita visualizar o tempo de espera restante e um botão de cancelamento do pedido.\\
\textbf{Severidade:} 2\\


\textbf{Problema 3:} Falta de consistência na interface \\
\textbf{Heurística:} H2-4: Consistência e adesão a normas\\
\textbf{Descrição:} Assim que se alterna entre o menu principal e o de uma funcionalidade, os botões que aparecem no ecrã à direita, alteram.\\
\textbf{Correção:} Optar por colocar os botões da funcionalidade noutro lugar.\\
\textbf{Severidade:} 2\\


\textbf{Problema 4:} Propicio a erros\\
\textbf{Heurística:} H2-5: Evitar erros\\
\textbf{Descrição:}   \\
\textbf{Correção:}  \\
\textbf{Severidade:} 3\\

\textbf{Problema 5:} Não é personalizavel \\
\textbf{Heurística:} H2-7: Flexibilidade e Eficiência\\
\textbf{Descrição:} Não permite personalizar a interface ao utilizador, nem existem aceleradores.\\
\textbf{Correção:} Criar uma conta de utilizador que guarde as preferencias.\\
\textbf{Severidade:} 1\\

\textbf{Problema 6:} Interface demasiado complexa\\
\textbf{Heurística:} H2-8: Desenho estético e minimalista\\
\textbf{Descrição:} Ecrã encontra-se demasiado cheio, o que pode dificultar a compreenção da mesma.\\
\textbf{Correção:} Implemantar uma interface mais minimalista. ??\\
\textbf{Severidade:} 3\\


\textbf{Problema 7:} Não permite repor definições \\
\textbf{Heurística:} H2-3: Utilizador controla e exerce livre arbitrio\\
\textbf{Descrição:} Não é possivel eleminar musicas ou restaurar a lista de musicas paras definiçoes de fabrico (lista vazia).\\
\textbf{Correção:} Criar uma opção que permita remover musica da playlist e para remover todas as musicas.\\
\textbf{Severidade:} 2\\



\end{document}