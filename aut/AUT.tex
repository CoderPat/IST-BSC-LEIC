\documentclass{article}
\usepackage[utf8]{inputenc}
\usepackage[top=1in, bottom=1.25in, left=1.25in, right=1.25in]{geometry}

\begin{document}
\section*{Análise de Utilizadores e Tarefas}

No âmbito do desenvolvimento do projeto, foi necessário identificar e caracterizar os utilizadores alvo, bem como os seus hábitos e necessidades. Assim para este fim, decidimos elaborar um pequeno questionário para responder às onze perguntas comuns que devem ser respondidas aquando da análise do utilizador de uma aplicação. Este relatório visa então expor os resultados do questionário e as respostas às onze perguntas.\\
O questionário era constituído por 25 perguntas e aquando da elaboração deste relatório, foi respondido por 91 pessoas.

\subsection*{Quem vai utilizar o sistema?}
Do que é possível retirar da primeira parte do questionário, o público alvo é constituído maioritariamente por jovens adultos com idades compreendidas entre os 18 e 29 anos (86\% dos inquiridos) com o grau de escolaridade referente ao ensino Secundário e Superior (58,24\% e 38,46\% respetivamente). A frequência de idas a bares situasse entre a 1 vez por ano e a 1 vez por semana
\subsection*{Que tarefas executam atualmente?}
Quando vão a um bar, os inquiridos demonstraram uma preferência por socializar (94,81\% dos inquiridos selecionaram esta opção), beber e comer (68,83\%) e ouvir música (57,14\%)
\subsection*{Que tarefas são desejáveis?}
Os inquiridos mostraram interesse em que a mesa barISTa tivesse funcionalidades como o sistema de votação para música ambiente (70.67\% do inquiridos), a sugestão automática de bebidas consoante os gostos do utilizador (69,33\%), um dispositivo de medição de álcool (65,33\%), uma coletânea de jogos de bebida (58,67\%) e capacidade de pedir cocktail personalizados (56\%).
\subsection*{Como se aprendem as tarefas?}
Aquando de aprender a utilizar a um sistema eletrónico novo (o que é o caso da mesa barISTa), 92,96\% dos questionados disse que aprendia por experiência de utilização, sendo que uma minoria (29,58\%) pede ajuda a terceiros.
\subsection*{Onde são desempenhadas as tarefas?}
Como seria de esperar, as tarefas desempenham-se num bar. Os inquiridos enquadram o ambiente típico de uma bar no centro das escalas de luminosidade e sonoridade (2,94 e 3,26 respetivamente numa escala de 1 a 5). Tipicamente fazem pedidos ao balcão (67,4\% dos inquiridos).
\subsection*{Qual a relação entre o utilizador e a informação?}
TODO: POR FAZER
\subsection*{Que outros instrumentos tem o utilizador?}
TODO: POR FAZER
\subsection*{Como comunicam os utilizadores entre si?}
TODO: POR FAZER
\subsection*{Qual a frequência de desempenho das tarefas?}
TODO: POR FAZER
\subsection*{Quais as restrições de tempo impostas?}
TODO: POR FAZER
\subsection*{O que acontece se algo correr mal?}
TODO: POR FAZER
\\

\section*{Funcionalidades a ser implementadas}

Da lista de funcionalidades propostas e mais votadas, selecionamos três que nos pareceram que seriam mais úteis e que tirariam mais proveito das capacidades da mesa interativa.

\subsection*{Cocktail personalizado}
Esta funcionalidade permite ao utilizador criar o um cocktail ao seu gosto, dizendo os ingredientes de uma lista pré definida.\\
\textbf{Cenário 1}: O José decide ir ao bar com uns amigos, mas quando lá chega e quer pedir o seu cocktail favorito, o Long Island, percebe que o bar não o tem no menu. Apesar disso, como o José tem conhecimento dos ingredientes, é capaz de o pedir na mesma, pois consegue descreve-lo através da mesa interativa.\\
\textbf{Cenário 2}: A Maria sugere ao Manel que experimente uma Margarita, mas o Manel é alérgico a laranja. Apesar disso, através da funcionalidade do Cocktail personalizado, ele é capaz de experimenta-la, substituindo o sumo de laranja por um de maça.
\\

\subsection*{Votação para Música Ambiente}
A votação para musica ambiente é um funcionalidade que permite ao utilizadores do o bar escolherem democraticamente que musica de fundo passa. Isso permite melhorar a experiência do utilizador pois podem ouvir musica que lhes agrade.\\
\textbf{Cenário 1}: A Joana vai a um bar que conhece com um grupo de amigos. Estão se a divertir, no entanto a música de fundo não os está a agradar muito. Como o bar não tem muita gente para alem deles e através da funcionalidade de votação, escolhem uma musica que lhes agrada mais, aproveitando melhor a experiência.\\
\textbf{Cenário 2}: TODO

\subsection*{Coletânea de jogos de bebida}
Uma coletânea de jogos de bebida disponíveis na mesa interativa permite aos utilizadores terem mais opções de diversão quando vão a um bar para além da tradicional socialização.\\
\textbf{Cenário 1}: Um grupo de amigo quer celebrar um aniversário, mas sente que o ambiente típico de um bar é demasiado calmo e não tem animação suficiente para um aniversário. Felizmente um deles conhece um bar com uma mesa barISTa, sendo que os jogos de bebida permitem tornar a noite mais interessante.\\
\textbf{Cenário 2}: TODO




\end{document}