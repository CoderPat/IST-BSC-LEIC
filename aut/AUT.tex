\documentclass{article}
\usepackage[utf8]{inputenc}
\usepackage[top=1.25in, bottom=1.25in, left=1.5in, right=1.5in]{geometry}

\begin{document}
\section*{Análise de Utilizadores e Tarefas}

No âmbito do desenvolvimento do projeto, foi necessário identificar e caracterizar os utilizadores alvo, bem como os seus hábitos e necessidades. Assim para este fim, decidimos elaborar um pequeno questionário para responder às onze perguntas comuns que devem ser respondidas aquando da análise do utilizador de uma aplicação. Este relatório visa então expor os resultados do questionário e as respostas às onze perguntas.\\
O questionário era constituído por 25 perguntas e aquando da elaboração deste relatório, foi respondido por 91 pessoas.

\subsection*{Quem vai utilizar o sistema?}
Do que é possível retirar da primeira parte do questionário, o público alvo é constituído maioritariamente por jovens adultos com idades compreendidas entre os 18 e 29 anos (86\% dos inquiridos) com o grau de escolaridade referente ao ensino Secundário e Superior (58,24\% e 38,46\% respetivamente). A frequência de idas a bares situasse entre a 1 vez por ano e a 1 vez por semana
\subsection*{Que tarefas executam atualmente?}
Quando vão a um bar, os inquiridos demonstraram uma preferência por socializar (94,81\% dos inquiridos selecionaram esta opção), beber e comer (68,83\%) e ouvir música (57,14\%)
\subsection*{Que tarefas são desejáveis?}
Os inquiridos mostraram interesse em que a mesa barISTa tivesse funcionalidades como o sistema de votação para música ambiente (70.67\% do inquiridos), a sugestão automática de bebidas consoante os gostos do utilizador (69,33\%), um dispositivo de medição de álcool (65,33\%), uma coletânea de jogos de bebida (58,67\%) e capacidade de pedir cocktail personalizados (56\%).
\subsection*{Como se aprendem as tarefas?}
Aquando de aprender a utilizar a um sistema eletrónico novo (o que é o caso da mesa barISTa), 92,96\% dos questionados disse que aprendia por experiência de utilização, sendo que uma minoria (29,58\%) pede ajuda a terceiros.
\subsection*{Onde são desempenhadas as tarefas?}
Como seria de esperar, as tarefas desempenham-se num bar. Os inquiridos enquadram o ambiente típico de um bar no centro das escalas de luminosidade e sonoridade (2,94 e 3,26 respetivamente numa escala de 1 a 5). Tipicamente fazem pedidos ao balcão (67,4\% dos inquiridos).
\subsection*{Qual a relação entre o utilizador e a informação?}
Com base nas respostas dadas, concluímos que a maioria(69,33\%) estaria interessada na existência de uma conta com as suas preferências, estando dispostos a partilhar algumas informações pessoais como o seu nome(73,33\%), gostos(66,67\%),e-mail e idade(cerca de 55\%).
\subsection*{Que outros instrumentos tem o utilizador?}
Quando questionados sobre os instrumentos utilizados na escolha do pedido, a maioria respondeu que já possuía ideias pré-definidas em relação ao seu pedido, aquando da ida ao bar. Logo, optam pela não utilização de instrumentos(62,34\%). No entanto, não tendo ideias pré-definidas, recorrem à carta/menu(57,14\%) ou à sugestão de conhecidos(50,65\%).
\subsection*{Como comunicam os utilizadores entre si?}
A maioria dos inquiridos comunica entre si pessoalmente(97,37\%), sendo que uma pequena minoria opta comunicar via eletrónica.
\subsection*{Qual a frequência de desempenho das tarefas?}
Relativamente à utilização de dispositivos eletrónicos, a maioria recorre aos mesmos várias vezes ao dia(71,43\%).
Num segundo plano, a frequência de idas a bares, pelos questionados, situasse entre 1 vez por ano e 1 vez por semana, sendo que a maioria opta por não conhecer pessoas novas.
\subsection*{Quais as restrições de tempo impostas?}
Os utilizadores têm preferência por um serviço de atendimento rápido(entre 1 a 5 minutos para que o seu pedido seja atendido). Para além disso, privilegiam um serviço operável entre 1 a 3 horas( a maioria dos utilizadores permanece entre 1 a 3 horas num bar).
\subsection*{O que acontece se algo correr mal?}
Deparados com uma incoerência do seu pedido, grande parte dos inquiridos opta por retornar o pedido e voltar a efetua-lo novamente(90,91\%). Quando o seu atendimento se prolonga mais que o normal, os utilizadores preferem avisar os empregados(55,84\%) ou então esperar (36,361\%).

\section*{Funcionalidades a ser implementadas}

Da lista de funcionalidades propostas e mais votadas, selecionamos três que nos pareceram que seriam mais úteis e que tirariam mais proveito das capacidades da mesa interativa.

\subsection*{Cocktail personalizado}
Esta funcionalidade permite ao utilizador criar o um cocktail ao seu gosto, dizendo os ingredientes de uma lista pré definida.\\
\textbf{Exemplo 1}: Cliente deseja um Long Island, bebida esta que o bar não tem no menu. Com esta funcionalidade o cliente consegue pedir um Long Island, a partir dos ingredientes disponibilizados na aplicação.\\
\textbf{Exemplo 2}: Pedir uma bebida semelhante a outra existente no menu em que a nova bebida pode possuir no máximo 2 ingredientes distintos da presente no menu. 
\\

\subsection*{Votação para Música Ambiente}
A votação para musica ambiente é um funcionalidade que permite ao utilizadores do o bar escolherem democraticamente que música de fundo passa. Isso permite melhorar a experiência do utilizador pois podem ouvir a música que lhes agrade.\\
\textbf{Exemplo 1}: Escolher uma música específica, pesquisando-a a partir do nome, caso esta faça parte do conjunto de músicas disponibilizadas pelo bar.\\
\textbf{Exemplo 2}: Escolher uma música de entre as cinco mais votadas até ao momento.\\

\subsection*{Coletânea de jogos de bebida}
Uma coletânea de jogos de bebida disponíveis na mesa interativa permite aos utilizadores terem mais opções de diversão quando vão a um bar para além da tradicional socialização.\\
\textbf{Exemplo 1}:Um grupo de pessoas,todas presentes na mesma mesa, jogarem um determinado jogo de bebida multi-player.\\
\textbf{Exemplo 2}: Uma pessoa sentada numa dada mesa desafia outra, sentada numa outra mesa, a jogar um dos jogos da coletânea, podendo assim ambos jogar na sua respetiva mesa.




\end{document}