\documentclass{article}
\usepackage[utf8]{inputenc}

\begin{document}
\section*{Análise de Utilizadores e Tarefas}

No âmbito do desenvolvimento do projeto, foi necessário identificar e caracterizar os utilizadores alvo, bem como os seus hábitos e necessidades. Assim para este fim, decidimos elaborar um pequeno questionário para responder às onze perguntas comuns que devem ser respondidas aquando da análise do utilizador de uma aplicação. Este relatório visa então expor os resultados do questionário e as respostas às onze perguntas.\\
O questionário era constituído por 25 perguntas e foi respondido por ?xx pessoas?.

\subsection*{Quem vai utilizar o sistema?}
TODO: POR FAZER
\subsection*{Que tarefas executam atualmente?}
TODO: POR FAZER
\subsection*{Que tarefas são desejáveis?}
TODO: POR FAZER
\subsection*{Como se aprendem as tarefas?}
TODO: POR FAZER
\subsection*{Onde são desempenhadas as tarefas?}
TODO: POR FAZER
\subsection*{Qual a relação entre o utilizador e a informação?}
TODO: POR FAZER
\subsection*{Que outros instrumentos tem o utilizador?}
TODO: POR FAZER
\subsection*{Como comunicam os utilizadores entre si?}
TODO: POR FAZER
\subsection*{Qual a frequência de desempenho das tarefas?}
TODO: POR FAZER
\subsection*{Quais as restrições de tempo impostas?}
TODO: POR FAZER
\subsection*{O que acontece se algo correr mal?}
TODO: POR FAZER
\\

\section*{Funcionalidades a ser implementadas}

Da lista de funcionalidades propostas e mais votadas, selecionamos três que nos pareceram que seriam mais úteis para um possível utilizador futuro da mesa interativa

\subsection*{Cocktail personalizado}
Esta funcionalidade permite ao utilizador criar o um cocktail ao seu gosto, dizendo os ingredientes de uma lista pré definida.\\
\textbf{Cenário 1}: O José decide ir ao bar com uns amigos, mas quando lá chega e quer pedir o seu cocktail favorito, o Long Island, percebe que o bar não o tem no menu. Apesar disso, como o José tem conhecimento dos ingredientes, é capaz de o pedir na mesma, pois consegue descreve-lo através da mesa interativa.
\textbf{Cenário 2}: A Maria sugere ao Manel que experimente uma Margarita, mas o Manel é alérgico a laranja. Apesar disso, através da funcionalidade do Cocktail personalizado, ele é capaz de experimenta-la, substituindo o sumo de laranja por um de maça.
\\

\subsection{Votação para Música Ambiente}
TODO: POR FAZER

\subsection{}


\end{document}