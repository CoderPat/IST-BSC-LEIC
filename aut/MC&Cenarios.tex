\documentclass{article}
\usepackage[utf8]{inputenc}
\usepackage[top=1.25in, bottom=1.25in, left=1.5in, right=1.5in]{geometry}

\begin{document}
\section*{Modelo Conceptual}



\subsection*{Metáfora}


\subsection*{Objetos, Atributos, Operações}

\subsection*{Relações entre conceitos}

\subsection*{Mapeamento}




\section*{Cenários de Atividades}

\subsection*{Cocktail personalizado}
\textbf{Cenário}: Após um dia de trabalho,o João decide ir a um bar beber uma copo. Quando chega ao bar, o João resolve  escolher um Long Island. Vai consultar a lista das bebidas, na aplicação barISTa, e apercebesse que esta não tem nenhum Long Island. Consequentemente, o João decide personalizar a sua bebida. A partir da lista de ingredientes disponibilizados pela funcionalidade Cocktail personalizado, o João seleciona, tequila, gin, whiskey, rum, vodka, Coca-Cola e sumo de laranja (ingredientes de um Long Island). Para cada ingrediente, ele escolhe a quantidade que deseja que esteja presente na sua bebida. Quando já não deseja adicionar mais nenhum ingrediente, o João finaliza o seu cocktail personalizado. No entanto, o João deseja pedir outra bebida para o seu amigo José que tinha acabado de chegar ao bar. O José quer uma caipirinha, bebida esta que consta na lista, só que em vez de limão quer morango. Então, o João decide personalizar a caipirinha. Assim, através da funcionalidade Cocktail personalizado, o João seleciona a caipirinha e altera o limão pelo morango(ingrediente disponibilizado pela funcionalidade), finalizando, de seguida, o cocktail. Dado que nenhum dos dois deseja pedir mais nada, o João submete o pedido. Quando o pedido de ambos chega, eles ficam bastante agradados pois as suas bebidas estão exatamente como desejavam.
\\

\subsection*{Votação para Música Ambiente}
\textbf{Cenário}: Dado que é sexta-feira à noite, o João decide ir a um bar para desanuviar de uma semana cansativa. Ao chegar ao bar, senta-se e faz o seu pedido. Enquanto espera pelo seu pedido vai apreciando a música ambiente até que decide ir votar na música ambiente. O João como é grande fã de Justin Bieber decide pesquisar pela canção Sorry do Justin Bieber. Como a canção é uma das músicas disponibilizadas pela funcionalidade Votação para Música Ambiente, o João, após a pesquisa, consegue votar na dita música. Passado alguns minutos o ele decide ir ver se a música em que tinha votado anteriormente, era uma das cinco mais votadas, de modo a ver se esta poderia ser das próximas a ser produzida. Vê que esta está em primeiro e passados uns minutos esta começa a dar no bar, ficando assim o João bastante contente. 

\subsection*{Coletânea de jogos de bebida}
\textbf{Cenário}:Após um dia de estudo intensivo o João, o José e o Pedro, decidem ir a um bar para se divertirem um pouco, ao chegarem ao bar apercebem-se que lhes apetece jogar um jogo de bebidas. Primeiramente escolhem as suas bebidas. Chegadas as bebidas, optam assim por uma partida com 3 jogadores. 
Passado algum tempo os 3 amigos, tendo ficado contentes com o jogo, decidem jogar uma segunda vez. No entanto decidem jogar com outros dois rapazes que tinham conhecido no bar, que estavam noutra mesa. Assim, convidam os dois rapazes para participarem no jogo com eles. 





\end{document}