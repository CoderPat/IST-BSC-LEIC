\documentclass[12pt]{article}
\usepackage[utf8]{inputenc}
\usepackage[top=2.5cm, left=3cm, right=3cm]{geometry}


\title{Análise da complexidade da solução para o melhor ponto de encontro}
\author{Patrick Fernandes\\81191 \and João Abrantes\\81845}
\date{Grupo 70}

\begin{document}
\maketitle
\section{Introdução}
O problema proposto consistia em, dado um certo número de localidades, filiais de uma empresa e caminhos entre localidades/filiais, com custos associados (considera-se que um custo negativo corresponde a um lucro) descobrir qual o ponto de encontro/localidade para as para carrinhas que partem das filiais se encontrar em que a perda total da empresa é minimizada.\\
\indent Tendo em conta os conceitos descritos é possível ver que é possível estabelecer uma analogia entre o cenário descrito e um grafo dirigido e pesado, onde as filiais e localidades são representadas por vértices, os caminhos entre elas as arestas e os custos associados a cada caminho como o peso da aresta. O problema pode então ser formulado em termos de grafos: dado um grafo e um conjunto de vértices F (filiais), qual o vértice do grafo cujo a soma das distancias dos vértices de F a esse vértice (tomando sempre o caminho mais curto) é mínima.
\section{A solução}
Ora analisando o problema, para descobrir o mínimo da distancia, tem de se calcular a distancia de todos os vértices que representam filiais a todos os vértices do grafo. Isto pode ser reduzido ao problema conhecido como \textbf{distancias mínimas entre todos os pares de vertices}. Para este problema podia ser usados dois algoritmos, o algoritmo de \textit{Floyd-Warshall} e o algoritmo de \textit{Johnson}, sendo que resolvem este problema. No entanto, pode-se reparar que o nosso caso em particular é um pouco mais simples, pois apenas requer-se a distancia de uma conjunto de vértices a todos os vértices. Ora, para o algoritmo de \textit{Floyd-Warshall} funcionar, é preciso calcular a distancia de todos os vértices a todos os vertices, enquanto que o algoritmo de \textit{Johnson}, que se baseia na aplicação do algoritmo de \textit{Dijkstra} a todos os vértices, pode ser alterado para apenas calcular o que nós queremos. Assim, para encontrar a distancia dos vértices das filiais a todos os vértices, aplica-se este último.\\
Após termos as distancias, para encontrar o vértice que pretendemos, basta para todos os vértices, somar-mos a distancias das filais até ele, e no final descobrir qual o vértice em que este valor é mínimo. Apesar de este problema ser bastante simplificado se usar-mos uma matriz FxL, isto pode causar problemas de memória para grandes inputs. Uma melhor maneira, é alterar o algoritmo de \textit{Johnson} para em vez de escrever numa matrix, escrever num vetor acumulador de dimensão L, que mantém a soma das distancias para qual o algoritmo de \textit{Dijkstra} já foi aplicado, sendo que no final o vertice com menor valor no acumulador é o melhor. No entanto, como perdemos os valores individuais das distancias dos filiais ao ponto de encontro no acumulador, podemos no final, inverter todas as arestas do grafo e aplicar mais uma iteração do algoritmo de \textit{Dijkstra}, ao vértice de encontro. Como o grafo foi invertido, a distancia desse vértice a uma filial corresponde a distancia de uma filial ao vértice no gafo original. \\
\indent Assim com esta sequencia de passos obtemos a solução para o problema original, sendo que o vértice representa o local de encontro, o seu valor no acumulador o custo de todos locais até ao ponto de encontro e a distancias calculadas na última iteração do algoritmo de \textit{Dijkstra} os custos individuais de cada filial ao ponto de encontro.
\section{Análise de Complexidade}
\subsection{Analise Teórica}
Designemos por $T(F, L ,E)$ a função que, dado o conjunto das filiais, de todos os locais e as ligações pesadas entre eles, devolve o tempo que o algoritmo demora a executar.
A maior parte do tempo de execução é passado na versão modificada do algoritmo de \textit{Johnson}. \\
\indent Neste, primeiro é executado o algoritmo de \textit{Bellman-Ford}, para repesar as arestas. Este algoritmo consiste no relaxamento de todas as aresta, num número de vezes igual aos vértices do grafo. Assim para o nosso grafo, a complexidade é $\mathcal{O}(|L||E|)$. \\
\indent Após a repesagem, já tendo o grafo com arestas positivas, são executadas $|F|$ vezes o algoritmo de \textit{Dijkstra}. Ora o algoritmo de \textit{Dijkstra} com Min-Heap para a fila de prioridades, tem complexidade de $\mathcal{O}(|E|log(|L|))$, pois o algoritmo olha para todas as aresta e no pior caso tem de atualizar uma distancia, que é $\mathcal{O}(log(|L|))$ numa heap. Ora executando o algorimto de \textit{Dijkstra} $|F|$ vezes, temos que a complexidade de calcular as distancias todas para somar ao acumulador é $\mathcal{O}(|F||E|log(|L|))$. No final tem de se inverter o grafo, que tem complexidade $\mathcal{O}(|L|+|E|)$, e calcular mais uma vez o algoritmo de \textit{Dijkstra} ($\mathcal{O}(|E|log(|L|))$) para as distancias individuais. Então a complexidade final é de \[ T(F, L ,E) = \mathcal{O}(|L||E|) + \mathcal{O}(|F||E|log(|L|)) + \mathcal{O}(|L|+|E|) + \mathcal{O}(|E|log(|L|))\] \\
\[ \Leftrightarrow  T(F, L ,E) = \mathcal{O}(|L||E| + |F||E|log(|L|))\]
\subsection{Analise Experimental}
Para confirmar a previsão teórica foi realizada uma analise experimental. Foram gerados 100 grafos com tamanhos compreendidos entre 1 e os 100000 vertices, esparsos (probabilidade de haver uma aresta entre dois vértices era de aproximadamente 1/V, sendo V o numero de vértices) e conexos, num programa Python com recurso à biblioteca NetworkX, e para cada grafo, foi medido o numero de instruções assembly (em modo de utilizador) usadas, que se pode assumir-se como unidade base de tempo.\\
\indent O algoritmo foi implementado em C++ usando vetores da STL e uma implementação nossa de uma Min-Heap, com as complexidades necessárias.
\end{document}
